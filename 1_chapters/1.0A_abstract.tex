The Murdoch University lolly machine is a demonstration unit designed to be used during open days with the intended purpose of showcasing the capabilities of \acrlong{icse} students at Murdoch University. The lolly machine sorts and dispenses lollies as a function of colour.  Prior to the completion of this thesis project, the machine was inoperable, and the control system was obsolete from a functional and maintenance perspective. The main objective of this thesis project,the \acrlong{lmu}, was to overhaul the existing control system, bringing it in line with modern technologies applicable to industry. 

Justification for the control system overhaul has two main arguments. The lolly machine will operate as intended and can be displayed and used on open days, and future students can use it as a project for continued development. 

The control system has been successfully overhauled and the machine is now in an operable state ready for future development. Multiple \acrlong{hmi}s have been developed across different user platforms allowing the device to be controlled through various methods. The machine can be controlled via a touch screen permanently installed on the device, a computer that must be physically connected to the machine or WiFi enabled devices. 

Unfortunately, the colour sensors do not work as intended and often fail to correctly register the lolly colour. The colour sensors currently installed on the machine need to be replaced - once this is complete the machine will be ready for use on open days. 