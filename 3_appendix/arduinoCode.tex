

\begin{lstlisting}[language=C, caption = Lolly machine arduino code., label=list:arduinoCode1]
#include "FastLED.h"
    // number of LEDs in the strip
    #define NUM_LEDS 60
    // define the data pin on the Arduino
    #define DATA_PIN 3
    // define the clock pin on the Arduino
    #define CLOCK_PIN 13
    // define variables
    long data;
    int bytesAtPort;
    byte r,g,b;
    int lollyCol = 1;
    int i;
    // create led object
    CRGB leds[NUM_LEDS];
    void setup() { 
          FastLED.addLeds<P9813, DATA_PIN, CLOCK_PIN, RGB>(leds, NUM_LEDS);
          Serial.begin(9600);
          Serial.setTimeout(500);
    }
    void loop() { 
    
      if (Serial.available() > 0) {
    	  
    	// check to see how many bytes are at port - diagnostics  
        bytesAtPort = Serial.available();
        
        // read the incoming byte:
        data = Serial.read();
        
        // say what you got:
        Serial.print("I received: ");
        Serial.println(data);
        // offset ascii value to get colour
        lollyCol = data -48;
        Serial.print("So the lolly colour number must be: ");
        Serial.println(lollyCol);
    
        switch(lollyCol){
          case 1: // white
            r = 0;
            g = 255;
            b = 255;
            break;
        
          case 2: // yellow/orange
            r = 255;
            g = 212;
            b = 0;
            break;
        
          case 3: // green
            r = 0;
            g = 255;
            b = 0;
            break;
        
          case 4: // pink
            r = 255;
            g = 0;
            b = 255;
            break;
        }
        
         // Turn the LED on, then pause
         for (i = 0; i < NUM_LEDS; i++){
         leds[i] = CRGB(r,g,b);
         }
         
         FastLED.show();
      }
    }
\end{lstlisting}

\begin{lstlisting}[language=C, caption=Arduino code that randomaly changes the clour of each RGB.  label=list:arduinoCode2]
    #include "FastLED.h"
    // How many leds in your strip?
    #define NUM_LEDS 60
    // For led chips like Neopixels, which have a data line, ground, and power, you just
    // need to define DATA_PIN.  For led chipsets that are SPI based (four wires - data, clock,
    // ground, and power), like the LPD8806 define both DATA_PIN and CLOCK_PIN
    #define DATA_PIN 3
    #define CLOCK_PIN 13
    // Define the array of leds
    long data;
    int bytesAtPort;
    byte r,g,b;
    int lollyCol = 1;
    int i;
    CRGB leds[NUM_LEDS];
    void setup() { 
          FastLED.addLeds<P9813, DATA_PIN, CLOCK_PIN, RGB>(leds, NUM_LEDS);
          Serial.begin(9600);
          Serial.setTimeout(500);
    }
    void loop() { 
      
    // frustrating christmas
      if (i > NUM_LEDS) {
        i = 1;
      }
    //  leds[i] = CRGB::Green;
    //  leds[i+1] = CRGB::Green;
    //  leds[i+2] = CRGB::Green;
    //  leds[i+3] = CRGB::Green;
    //  leds[i-1] = CRGB::Red;
    //  i++;
    //  delay(i*10)
    
      
    // fairy display
      r = random(1,255);
      g = random(1,255);
      b = random(1,255);
      i = random(0,59);
      leds[i] = CRGB(r,g,b);
    
      delay(500);
      
      FastLED.show();
      
      }
    
\end{lstlisting}